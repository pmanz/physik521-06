% Für Seitenformatierung

\documentclass[DIV=15]{scrartcl}

% Zeilenumbrüche

\parindent 0pt
\parskip 6pt

% Für deutsche Buchstaben und Synthax

\usepackage[ngerman]{babel}

% Für Auflistung mit speziellen Aufzählungszeichen

\usepackage{paralist}

% zB für \del, \dif und andere Mathebefehle

\usepackage{amsmath}
\usepackage{commath}
\usepackage{amssymb}

% für nicht kursive griechische Buchstaben

\usepackage{txfonts}

% Für \SIunit[]{} und \num in deutschem Stil

\usepackage[output-decimal-marker={,}]{siunitx}
\usepackage[utf8]{inputenc}

% Für \sfrac{}{}, also inline-frac

\usepackage{xfrac}

% Für Einbinden von pdf-Grafiken

\usepackage{graphicx}

% Umfließen von Bildern

\usepackage{floatflt}

% Für Links nach außen und innerhalb des Dokumentes

\usepackage{hyperref}

% Für weitere Farben

\usepackage{color}

% Für Streichen von z.B. $\rightarrow$

\usepackage{centernot}

% Für Befehl \cancel{}

\usepackage{cancel}

% Für Layout von Links

\hypersetup{
	citecolor=black,
	colorlinks=true,
	linkcolor=black,
	urlcolor=blue,
}

% Verschiedene Mathematik-Hilfen

\newcommand \e[1]{\cdot10^{#1}}
\newcommand\p{\partial}

\newcommand\half{\frac 12}
\newcommand\shalf{\sfrac12}

\newcommand\skp[2]{\left\langle#1,#2\right\rangle}
\newcommand\mw[1]{\left\langle#1\right\rangle}

\renewcommand \exp[1]{\mathrm{exp}\del{#1}}

% Nabla und Kombinationen von Nabla

\renewcommand\div[1]{\skp{\nabla}{#1}}
\newcommand\rot{\nabla\times}
\newcommand\grad[1]{\nabla#1}
\newcommand\laplace{\triangle}
\newcommand\dalambert{\mathop{{}\Box}\nolimits}

%Für komplexe Zahlen

\newcommand \ii{\mathrm i}
\renewcommand{\Im}{\mathop{{}\mathrm{Im}}\nolimits}
\renewcommand{\Re}{\mathop{{}\mathrm{Re}}\nolimits}

%Für Bra-Ket-Notation

\newcommand\bra[1]{\left\langle#1\right|}
\newcommand\ket[1]{\left|#1\right\rangle}
\newcommand\braket[2]{\left\langle#1\left.\vphantom{#1 #2}\right|#2\right\rangle}
\newcommand\braopket[3]{\left\langle#1\left.\vphantom{#1 #2 #3}\right|#2\left.\vphantom{#1 #2 #3}\right|#3\right\rangle}


\setcounter{section}{0}
\renewcommand\thesection{H\,6.\arabic{section}}
\renewcommand\thesubsection{\thesection.\alph{subsection}}

\title{physik521: Übungsblatt 06}
\author{%
    Lino Lemmer \\ \small{\texttt{s6lilemm@uni-bonn.de}}
    \and
    Martin Ueding \\ \small{\texttt{mu@martin-ueding.de}}
    \and
    Paul Manz \\ \small{\texttt{p.m@uni-bonn.de}}
}

\begin{document}
\maketitle

\section{Reißverschlussmodell für DNA-Moleküle}
\subsection{}
Das betrachtete System hat eine konstante Teilchenzahl und befindet sich im Wärmeaustausch mit einem Reservoir. Wir können es deshalb als kanonisches Ensemble auffassen. Die kanonische Zustandssumme ist

\begin{align*}
Z_k = \sum_n \exp{-\frac{E_n}{k_B T}} \\
\end{align*}
Für die Energie gilt dabei einfach $E_n=n\epsilon$, wobei $n$ die Anzahl der geöffneten Bindungen ist. Also:
\begin{align*}
Z_k = \sum_n \exp{-\frac{n\epsilon}{k_B T}} \\
\end{align*}

\subsection{}
Mit
\begin{align*}
W_k(n) &= \frac{1}{Z_k} \exp{-E_n \beta} \\
&= \frac{1}{Z_k} \exp{-\frac{n \epsilon}{k_B T}}
\end{align*} 
gilt für die mittlere Zahl der offenen Bindungen  $\langle n \rangle $:
\begin{align*}
\langle n \rangle &= \sum_n n W_k(n) \\
&= \frac{1}{Z_k} \sum_n n \ \exp{-\frac{n \epsilon}{k_B T}} \\
&= \frac{\sum_n n \ \exp{-\frac{n \epsilon}{k_B T}}}{\sum_n \exp{-\frac{n\epsilon}{k_B T}}}
\end{align*}
Definiere $x := \exp{\epsilon \beta}$:
\begin{align*}
\langle n \rangle &= \frac{\sum_n^{N-1} n x^n}{\sum_n^{N-1} x^n} \\
&=\frac{\frac{Nx^{N+1}-Nx^{N}+x}{(x-1)^2}}{\frac{1-x^{N}}{1-x}} \\
&=\frac{Nx^{N+1}-Nx^{N}+x}{(x-1)(x^{N}-1)} \\
&=\frac{Nx^{N+1}-Nx^{N}+x}{x^{N+1}-x^{N}-x+1}
\end{align*}

\subsection{}
Zu bestimmen ist der Anteil offener Bindungen $\langle n \rangle/ N$ im Limes $N \rightarrow \infty$ als Funktion von $x$.
\begin{align*}
\lim_{N \rightarrow \infty} \frac{\langle n \rangle}{N}=\lim_{N \rightarrow \infty} \frac{x^{N+1}-x^{N}+x/N}{x^{N+1}-x^{N}-x+1}
\end{align*}
Wir betrachten zunächst den Fall $x<0$:
\begin{align*}
\lim_{N \rightarrow \infty} \frac{x^{N+1}-x^{N}+x/N}{x^{N+1}-x^{N}-x+1}=\frac{0}{1-x}=0 \\
\end{align*}
Nun den Fall $x>0$:
\begin{align*}
\lim_{N \rightarrow \infty} \frac{x^{N+1}-x^{N}+x/N}{x^{N+1}-x^{N}-x+1}= \lim_{N \rightarrow \infty} \frac{x^{N+1}-x^{N}}{x^{N+1}-x^{N}}=1
\end{align*}
Ein sehr langer DNA-Strang ist demnach, abhängig von der Temperatur entweder vollständig geschlossen oder vollständig aufgespalten.

\section{Zustandsdichte}
\subsection{Elektronen}
Für (nichtrelativistische) Elektronen gilt die folgende Dispersionsrelation:
\[\epsilon (p) = \frac{p^2}{2m} \]
Damit ergibt sich für die Zustandsdichte für einen eindimensionalen Kasten:
\begin{align*}
\mathcal{N}\del{\epsilon} &= \frac{1}{2\pi\hbar} \int \dif p \  \delta\del{\epsilon - \frac{p^2}{2m}} \\
&= \frac{1}{2\pi\hbar} \int \dif q \frac{m}{\sqrt{q}} \ \delta \del{\epsilon-q} \\
&= \frac{m}{2\pi\hbar\sqrt{\epsilon}} \\
\end{align*}
Für $d=2$:
\begin{align*}
\mathcal{N}\del{\epsilon} &= \frac{1}{(2\pi\hbar)^2} \int \dif^2 p \ \delta \del{\epsilon-\frac{p^2}{2m}} \\
&=\frac{1}{(2\pi\hbar)^2} \int \dif \phi \int \dif p \ p \ \delta \del{\epsilon-\frac{p^2}{2m}} \\
&=\frac{2\pi}{(2\pi\hbar)^2} \int \dif q \ m \ \delta(\epsilon-q) \\
&=\frac{m}{2\pi\hbar^2} \\
\end{align*}
Für $d=3$:
\begin{align*}
\mathcal{N}\del{\epsilon} &= \frac{1}{(2\pi\hbar)^3} \int \dif^3 p \ \delta \del{\epsilon-\frac{p^2}{2m}} \\
&= \frac{1}{(2\pi\hbar)^3} \int \dif \Omega \int \dif p \ p^2 \ \delta \del{\epsilon-\frac{p^2}{2m}} \\
&= \frac{4\pi}{(2\pi\hbar)^3} \int \dif p \ p^2 \ \delta\del{\epsilon-\frac{p^2}{2m}} \\
&= \frac{1}{2\pi^2\hbar^3} \int \dif q \sqrt{q} m \ \delta(\epsilon-q) \\
&= \frac{m \sqrt{\epsilon}}{2\pi^2\hbar^3}
\end{align*}

\subsection{Photonen}
Für Photonen gilt die Dispersionsrelation $\epsilon(p)=cp$. Damit gilt für die Zustandsdichte für $d=1$:
\begin{align*}
\mathcal{N}\del{\epsilon} &= \frac{1}{2\pi\hbar} \int \dif p\ \delta(\epsilon - cp) \\
&=\frac{1}{2\pi\hbar c}
\end{align*}
Für $d=2$:
\begin{align*}
\mathcal{N}\del{\epsilon} &= \frac{1}{(2\pi\hbar)^2} \int \dif \phi \int \dif p \ p \ \delta(\epsilon-cp) \\
&=\frac{2\pi}{(2\pi\hbar)^2} \int \dif q \ \frac{q}{c^2} \  \delta(\epsilon-q) \\
&=\frac{\epsilon}{2\pi\hbar^2 c^2}
\end{align*}
Für $d=3$:
\begin{align*}
\mathcal{N}\del{\epsilon} &= \frac{1}{(2\pi\hbar)^3} \int \dif \Omega \int \dif p \ p^2 \delta(\epsilon-cp) \\
&= \frac{1}{2\pi^2\hbar^3} \int \dif q \ \frac{q^2}{c^3} \  \delta(\epsilon-q) \\
&=\frac{\epsilon^2}{2\pi^2\hbar^3c^3}
\end{align*}

\subsection{Relativistische Teilchen}
Für relativistische Teilchen gilt:
\begin{align*}
\epsilon(p)^2 &= (pc)^2+(mc^2)^2 \\
\implies \epsilon(p) &= \sqrt{(pc)^2+(mc^2)^2}
\end{align*}
Dann gilt für die Zustandsdichte bei $d=1$:
\begin{align*}
\mathcal{N}\del{\epsilon} &= \frac{1}{2\pi\hbar} \int \dif p \ \delta \del{\epsilon - \sqrt{(pc)^2+(mc^2)^2}} \\
&= \frac{1}{2\pi\hbar} \int \dif q \ \frac{q}{c\sqrt{q^2-(mc^2)^2}} \delta \del{\epsilon-q} \\
&= \frac{\epsilon}{2\pi\hbar c\sqrt{\epsilon^2-(mc^2)^2}}
\end{align*}
Für $d=2$: 
\begin{align*}
\mathcal{N}\del{\epsilon} &= \frac{1}{(2\pi\hbar)^2} \int \dif \phi \int \dif p \ p \ \delta \del{\epsilon - \sqrt{(pc)^2+(mc^2)^2}} \\
&= \frac{1}{2\pi\hbar^2} \int \dif q \ \frac{q}{c^2} \\
&= \frac{1}{2\pi\hbar^2 c}
\end{align*}
Für $d=3$:
\begin{align*}
\mathcal{N}\del{\epsilon} &= \frac{1}{(2\pi\hbar)^3} \int \dif \Omega \int \dif p \ p^2 \ \delta \del{\epsilon - \sqrt{(pc)^2+(mc^2)^2}} \\
&= \frac{1}{2\pi^2\hbar^3} \int \dif q \frac{q\sqrt{q^2-(mc^2)^2}}{c^3} \ \delta(\epsilon-q) \\
&= \frac{\epsilon\sqrt{\epsilon^2-(mc^2)^2}}{2\pi^2\hbar^3 c^3}
\end{align*}
\end{document}
